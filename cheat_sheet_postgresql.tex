\documentclass[%
	11pt,
	a4paper,
	utf8,
	%twocolumn
		]{article}	

\usepackage{style_packages/podvoyskiy_article_extended}


\begin{document}
\title{Наиболее полезные конструкции PostgreSQL}

\author{}

\date{}
\maketitle

\thispagestyle{fancy}

\tableofcontents



% Источники в "Газовой промышленности" нумеруются по мере упоминания 
\begin{thebibliography}{99}\addcontentsline{toc}{section}{Список литературы}
	\bibitem{chacon:2020}{ \emph{Чакон С.}, \emph{Штрауб Б.} Git для профессионального программиста. -- СПб.: Питер, 2020. -- 496~с. }
	
	\bibitem{sobel:2011}{ \emph{Собель М}. Linux. Администрирование и системное программирование. 2-е изд. -- СПб.: Питер, 2011. -- 880 с. }
\end{thebibliography}

%\listoffigures\addcontentsline{toc}{section}{Список иллюстраций}

\end{document}
